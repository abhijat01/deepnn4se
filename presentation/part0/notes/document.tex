\documentclass[14pt, twocolumn]{article}
\usepackage[margin=0.5in]{geometry}
\usepackage{hyperref}
%opening
\title{}
\author{}

\begin{document}

\maketitle


\section*{Introduction}
Thanks for the feedback. Every single feedback response requested more information on the topic
hence here we are.
\begin{itemize}
	\item Lack of blackboard for math compounded with remote presentation - feel free to ask 
	questions. Ambitious agenda so I might not be able to get to all questions but will try. 
	\item  The group is comfortable with linear algebra, multi-variate calculus, gradients 
	and optimization in multiple dimensions etc. Very little linear algebra in the presentation 
	but you will need it. 
	\item What will not be covered - refer to the slide. 
	\item Objective - you should feel like can start playing with python code now.
\end{itemize}

\section*{Convolution network}
Discuss why architecture matters.
\begin{itemize}
	\item Exploit relationships better 
	\item Numerical stability/ease of computation.
\end{itemize}

\subsection*{ImageNet challenge}
\begin{itemize}
	\item Describe the challenge 
	\item Start taking about what happened in 2012 
\end{itemize}


\section*{Transfer learning}
\subsection*{Motivation}
\begin{itemize}
	\item People not only build models but also make the trained weights available. 
	We will see a little bit of pytorch code. Data, fine tuning and execution all 
	reused. 
	\item We may not have enough data to train a deep model. 
	\item Our task can exploit features detected by an existing pre-trained model. 
\end{itemize}

\section*{Adversarial example}
\begin{itemize}
	\item Start with Rohan's work 
	\item Introduce adversarial network 
	\item Discuss stop sign, trucks with LCDs 
\end{itemize}

\nocite{dumoulin2018guide}
\nocite{726791}
\nocite{43405}
\nocite{45818}
\nocite{10.5555/3086952}
\nocite{NIPS2014_5346}
\nocite{russell2016artificial}
\bibliography{software-engineering-ref} 
\bibliographystyle{ieeetr}
\end{document}
