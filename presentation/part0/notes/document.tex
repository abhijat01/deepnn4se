\documentclass[14pt, twocolumn]{article}
\usepackage[margin=0.5in]{geometry}
\usepackage{hyperref}
%opening
\title{}
\author{}

\begin{document}

\maketitle


\section*{Introduction and plan}

I will be going back and forth between multiple decks. I will also try a live demo 
which as you know, will certainly not work the one time I will want it to. There is an added problem of 
Teams and how it handles changes to the screen. So a few glitches but please bear with me. 


Lack of blackboard for math compounded with remote presentation - feel free to ask 
	questions. Ambitious agenda so I might not be able to get to all questions but will try. w.


\section*{Objective}
The objective of this session is not to make you experts in AI or machine learning or deep neural 
networks. This is just to give you a flavor, an idea of what people mean when they use these terms. 
I will briefly address data issue but I strongly suspect that I won't be doing it justice. 

I do not have an end point. There are a few items that I do wish to cover. I will let the discussion and drinks drive the agenda beyond that. 

\section*{Notes about sections}
\begin{description}
	\item[Support vector machines] I was all into support vector machines in 2008 - 2009. Neural networks were 
	relegated to the backwaters of machine learning community. 
	\item[Logistic planning] 1991 Gulf war crisis - US forces deployed Dynamic analysis and replanning tool (DART) 
	for transportation. They estimated that this single instance more than paid for DARPA's 30-year investment in 
	AI. 
	\item[Graphs] Talk about max flow and min cut problem.
\end{description}

\section*{Convolution network}
Discuss why architecture matters.
\begin{itemize}
	\item Exploit relationships better 
	\item Numerical stability/ease of computation.
\end{itemize}

\subsection*{ImageNet challenge}
\begin{itemize}
	\item Describe the challenge 
	\item Start taking about what happened in 2012 
\end{itemize}


\section*{Transfer learning}
\subsection*{Motivation}
\begin{itemize}
	\item People not only build models but also make the trained weights available. 
	We will see a little bit of pytorch code. Data, fine tuning and execution all 
	reused. 
	\item We may not have enough data to train a deep model. 
	\item Our task can exploit features detected by an existing pre-trained model. 
\end{itemize}

\section*{Adversarial example}
\begin{itemize}
	\item Start with Rohan's work 
	\item Introduce adversarial network 
	\item Discuss stop sign, trucks with LCDs 
\end{itemize}

\nocite{dumoulin2018guide}
\nocite{726791}
\nocite{43405}
\nocite{45818}
\nocite{10.5555/3086952}
\nocite{NIPS2014_5346}
\nocite{russell2016artificial}
\bibliography{software-engineering-ref} 
\bibliographystyle{ieeetr}
\end{document}
