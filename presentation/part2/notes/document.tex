\documentclass[14pt, twocolumn]{article}
\usepackage[margin=0.5in]{geometry}
\usepackage{hyperref}
%opening
\title{}
\author{}

\begin{document}

\maketitle


\section*{Introduction}
\begin{itemize}
	\item Lack of blackboard for math compounded with remote presentation - feel free to ask 
	questions. Ambitious agenda so I might not be able to get to all questions but will try. 
	\item  The group is comfortable with linear algebra, multi-variate calculus, gradients 
	and optimization in multiple dimensions etc. Very little linear algebra in the presentation 
	but you will need it. 
	\item What will not be covered - refer to the slide. 
	\item Objective - you should feel like can start playing with python code now.
\end{itemize}
\section*{Using vectors}
\begin{itemize}
	\item This is basic matrix multiplication with vectors. Not very interesting. 
	\item Move on to gradients. Point out that there is nothing special, you just 
	work with elements and try to step back to matrices and see if your notations 
	make sense. Quite possible that you will have totally different notation from 
	others and it will still work! But then you will need to explain it better.
	\item Element wise - nothing special. We will come across sigmoid again. 
\end{itemize}

\section*{Auto-diff}
Small models and tens of thousands of variables. You will very quickly reach 
100s of thousands to million. BERT base has 110 million and BERT large has 
340 million parameters. GPT-3 has a whopping 175 billion parameters. 
\begin{itemize}
	\item Analytical approach is impossible for all practical purposes. 
	\item Finite difference will be too slow with too many numerical issues. 
	\item Auto-diff works because it is something in-between analytical and 
	numeric. You break your computation into layers. For each layer, you implement 
	the function that the layer computes. Since the layer is small, you can actually 
	also compute the analytical derivative. You use the analytical expression to 
	compute the value of gradient. There are still problems with it is still better. 
	\item Reverse mode exploits chain rule in differentiation. It starts computation 
	from the end and reaches all the starting points (parameters) in a single pass (
	excluding the forward pass).  
\end{itemize}
Before we can use auto-diff, we represent the computation as a graph. This is a pretty 
standard idea in computer science. We can use this to even implement symbolic 
differentiation (contact me if you wish to build an engine just for fun). 

It is important  that you follow the graph carefully. 

\begin{itemize}
	\item Just follow Node 1 and Node 2 to Node 7
	\item Point out that each node can be implemented independent of others. We 
	can compute the function as well as the derivative (gradients) of inputs W.R.T
	to output. 
	\item On the analytical slide, point out computation for z. {\bf Do not spend 
	time on other computations}
	\item Show the example of multiplication node and gradients. Point out that 
	the code is for illustrative purposes only and was written solely for this 
	presentation.
\end{itemize}

\section*{Fit}
\begin{itemize}
	\item For sotchastic gradient  descent - we select a batch, compute the gradient 
	of parameters with respect to loss, update the parameters, then move on to
\end{itemize}

\end{document}
